\section{Introduction}

For many reasons, the need for communication has become one of the most urgent
problems in the modern world. Nowadays, smartphones are almost universally used
as the primary form of communication. Consequently, it would be highly
advantageous if individuals could communicate using smartphones in the absence
of infrastructure. This would eliminate their dependence on infrastructure and
enable them to communicate without infrastructure or provide an alternative to
available infrastructure. There are many well-known smartphone chat
applications, including WhatsApp, Telegram, Viber, and others;
nevertheless, they all depend on the internet and the appropriate communication
infrastructure.

The main infrastructure-based communication methods in smartphones are
cell-based GSM, 4G LTE, and access point-based WiFi, and assorted technologies.
But when it comes to an infrastructure-less environment, the nodes do not have
to connect to the
cellular towers since they establish direct wireless connections. In this case,
the ability to leverage various infrastructure-free communication protocols is
referred to as being adaptive. Thus, mobile ad-hoc networks are networks
created dynamically without the need for infrastructure, presuming that all
nodes connecting to the network will be portable to some
extent\cite{chlamtac2003}.

Wi-Fi (in ad-hoc mode) and Bluetooth come out on top regarding available
infrastructure-less communication methods. Bluetooth is one of the most popular
short-range wireless technologies for transferring data between devices over
short distances. It took the place of the conventional cable connections that
were used to transport data between our regularly used portable gadgets. When
compared to Wi-Fi, Bluetooth is less expensive and uses less power, but Wi-Fi
offers more benefits, such as faster speeds and broader transmission ranges.
Although Wi-Fi's ad hoc mode could handle direct device-to-device (D2D)
communication without needing an Access Point (AP), this requires rooting of
the device in Android smartphones\cite{soares2017}. On the other hand, Wi-Fi
Direct (Wi-Fi P2P), which is
constructed on top of the infrastructure mode of Wi-Fi, enables simple and
effective D2D communication and lets the devices choose who will take over the
AP-like functions. Another technology to consider is Bluetooth Low Energy
(BLE), a development of Bluetooth that significantly reduces battery usage
because it establishes brief connections and stays in sleep mode when not
connected.

Factors that need to be considered when developing a framework for
infrastructure-less communication in mobile devices are affordable power
consumption, message transmission dependability, message delays, routing
mechanism performance, and network establishment speed in
different scenarios. Also, the topology of mobile ad
hoc networks frequently changes due to factors including node mobility and
unexpected node unavailability.

Due to this, it is crucial to find appropriate routing methods and neighbor
identification techniques to increase communication effectiveness in these
kinds of networks.

Meshify is a framework for infrastructure-less communication using smartphones.
It is currently equipped only with Bluetooth classic and BLE
technologies\cite{gunasekara2022}. Implementing Wi-Fi Direct as an additional
wireless communication technology for Meshify will significantly enhance its
connectivity capabilities. Wi-Fi Direct offers faster data transfer speeds and
a more stable connection than Bluetooth, making it ideal for real-time data
exchange applications, as explained above. This expansion will empower Meshify
with a versatile and robust communication framework, improving its performance
and usability. Also, improvements should focus on improving the current Meshify
neighbor discovery protocol for accurate device detection and multi-hop message
relaying to extend network reach. Incorporating research from opportunistic
networks can enhance communication resilience. In addition, the Meshify
middleware framework should adhere to the best software engineering principles
and design patterns to ensure that third-party app developers may readily use
it.
