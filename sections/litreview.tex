\section{Literature review}

The core feature that sets ad hoc networks apart from conventional networks is
the need for no infrastructure to create and maintain the network. In an ad hoc
network, the nodes act as network relays, eliminating the need for established
infrastructure. Due to these factors, ad hoc networks are extensively used in
sensor networks and other IoT-related domains\cite{akyildiz2002}. Another
instance where infrastructure-less networks become useful is in times of
disaster and defense scenarios, where existing infrastructure may be
unavailable or unreliable. In such cases, ad hoc networks can aid in
maintaining networks confined to vehicles and other mobility units and can aid
in critical communications in disaster scenarios, even forming part of a
corporate data backup system\cite{raza2016}.  However, in recent years,
consumer use of smartphones skyrocketed. The global smartphone usage is
estimated to be 6.92 billion, roughly 86\% of the world population in
2023\cite{zippia2023}. This availability makes smartphones an ideal device for
consumer-level general-purpose ad hoc networking\cite{soares2017}.

Since the foremost need among users of mobile devices is
connectivity\cite{chlamtac2003}, mobile devices in self-configured,
self-maintained networks delivering network connectivity where no connection
infrastructure is present is highly desirable\cite{chlamtac2003}. Apart from
intra-network communication, Mobile Ad Hoc Networks can also share internet
service via an access point, effectively extending the Internet service
area\cite{chlamtac2003}.

While the use of ad hoc networks for IoT and sensor-based applications is
well-established, the same implementations can not be applied to consumer
mobile devices due to limitations imposed by the operating systems and network
implementations of mobile devices.

\subsection{Ad Hoc Networking Frameworks for Android}

This section aims to establish the current state of the art on Android
frameworks and implementations aiming to facilitate ad hoc network
capabilities.

The Serval Project introduces an application that creates self-managed meshable
networks using the Wi-Fi interfaces of the devices\cite{gardner2013}. The
Serval system also supports a mesh extender\cite{gardner2013}. This
air-droppable UHF packet radio extends the mesh network's service range, making
it suitable for providing stable networks where infrastructure is not present -
extending the initial use case of disaster management. However, since the
Serval mesh is a standalone application and is not a framework available to use
by programmers.

Bluetooth and Bluetooth Low-Energy connectivity are also used to create
proximity-based communication networks. Wang et al. propose a  framework and a
proof of concept chat application using Bluetooth and
Wi-Fi-direct\cite{wang2015}. Since each device carries two unique identifiers
for Bluetooth and Wi-Fi-direct, the system uses a device serial number for
identification. The routing is done with a simple flood-based routing protocol,
and each node reads and floods each received message\cite{wang2015}. Therefore,
the accompanying proof of concept application does not support private
messages. Moreover, the framework assumes that Bluetooth master devices and
WiFi-Direct Group Owners are up and available before the clients are connected.
Therefore, the framework is not easy to use in real-world scenarios as in
MANETs due to node mobility, the topology changes rapidly, and in the case of
WiFi Direct, if the group owner is disconnected, the group will be discarded
and recreated\cite{raza2016}. The framework proposed by Wang et al. is not
resilient to these conditions\cite{wang2015}.

Meshrabiya\cite{meshrabiya} is another framework utilizing the core networking
features of the Android operating system. The framework uses the Wi-Fi Direct
(WiFiDi) standard to create and maintain an ad hoc network with
WiFiDi-compliant and legacy devices. The framework achieves this by maintaining
multiple groups with fully qualified Group Owners and legacy hotspot mode
devices. For routing, the framework uses the link-local IPV6 addresses.
However, as mentioned in the framework documentation\cite{meshrabiya}, there are
limitations in devices running Android versions higher than 11 due to the
access point concurrency feature. Moreover, since the only communication method
available in the framework is WiFi, the usability of the framework is reduced.
Also, another notable factor is that the implementation requires a QR code to
be scanned for inter-group routing.  A more modular framework where the
application programmers can select the communication method as needed will
increase the framework's usability.

Bridgefy\cite{bridgefy} is another framework implemented to be a
programmer-friendly implementation of mesh networks using Android smartphones.
While the Bridgefy framework supports iOS and Android devices with universal
compatibility, the framework only employs Bluetooth as the communication
technology.

The above findings are summarized in table~\ref{litreviewMatrix} for ease of reference.

As evident by the literature review, there is ample opportunity for developing
a general application framework that allows programmers to access networks in a
generalized manner. Since the framework must be application or
scenario-agnostic, the routing algorithms must be chosen based on a generalized
mobility model.

\begin{table}[htbp]
    \centering
    \begin{tabularx}{\textwidth}{|X|X|X|X|X|X|X|}
        \hline
        \textbf{Reference} & \textbf{Accessible Framework} & \textbf{Autonomous Setup and Repair} & \textbf{Open Source} & \textbf{WiFi-Direct} & \textbf{Bluetooth or BLE} & \textbf{Rooting or modification to OS is not needed}\\ \hline
        Meshify\cite{gunasekara2022} & \checkmark & \checkmark & \checkmark & \ding{55} & \checkmark & \checkmark \\ \hline
        Bridgefy\cite{bridgefy} & \checkmark & \checkmark & \ding{55} & \ding{55} & \checkmark & \checkmark \\ \hline
        Meshrabia\cite{meshrabiya} & \checkmark & \ding{55} & \checkmark & \checkmark & \ding{55} & \checkmark \\ \hline
        Serval Project\cite{gardner2013} & \ding{55} & \checkmark & \checkmark & \ding{55} & \checkmark & \checkmark \\ \hline
        Wang et al.\cite{wang2015} & \ding{55} & \ding{55} & \ding{55} & \checkmark & \checkmark & \checkmark \\ \hline
        \textbf{This Project} & \checkmark & \checkmark & \checkmark & \checkmark & \checkmark & \checkmark \\ \hline 
    \end{tabularx}
    \caption{Comparison of existing frameworks}
    \label{litreviewMatrix}
\end{table}

\subsection{Routing Protocols}

Proactive routing protocols that seek out nodes and maintain a routing table
are not suitable for routing ad hoc networks due to the highly mobile nature of
the nodes. As nodes move in and out of the network coverage area, the network's
topology is altered, outdating the gathered routing information. Therefore, the
suitable routing protocols are reactive\cite{reina2011}. Raza et
al.\cite{raza2016} analyze several different protocols for different network
traffic and connection types using metrics such as throughput and end-to-end
delay.

The following sections explain some widely studied routing protocols in ad hoc
networks.

\subsubsection{Simple Flooding}

Flooding is one of the most basic routing methods used in MANETs since
broadcast forwarding is the main method of operation in many wireless
networks\cite{chlamtac2003}. In MANETs, this involves nodes forwarding every
incoming packet to all the neighbors within its range. However, this is not
ideal as with the increase of the size of the network, flooding can create
broadcast storms and decrease network availability\cite{chlamtac2003}.

\subsubsection{Ad hoc Demand Distance Vector (AODV) }

AODV is a reactive routing protocol allowing unicast routing. When a node needs
to send a packet to another node, it sends an RREQ packet mentioning the
required destination node\cite{reina2011}. The neighboring nodes forward the
RREQ packet until it reaches the destination node, and RREP is sent by the
destination node and is doubled back to the source, enabling the source to
route the data packet to the destination node\cite{reina2011}.

\subsubsection{Dynamic Source Routing (DSR)}

In DSR, instead of depending on the nodes to handle routing, the complete path
to the destination node is sent with the RREP packet in the header to the
source node\cite{chlamtac2003}.

As found by Reina et al. with simulated node mobility models, AODV performs
better than other routing protocols\cite{reina2011}.
